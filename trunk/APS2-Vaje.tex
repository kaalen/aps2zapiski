\documentclass[a4paper,10pt]{scrartcl}
\usepackage[utf-8]{inputenc}
\usepackage[slovene]{babel}
\usepackage{graphicx}
\usepackage{hyperref}
\usepackage{listings}
\usepackage{epsfig}
\lstset{numbers=left, frame=single}
\usepackage{amsmath}
\usepackage{amsfonts}
\usepackage{amssymb}
\usepackage{float}
\usepackage{floatflt}

%opening
\title{Skripta za vaje pri predmetu Algoritmi in podatkovne strukture 2}
\author{Caserman Alenka}

\begin{document}

\maketitle

\tableofcontents

\part{Urejanje}

\section{Notranje urejanje}

\subsection{Hitro urejanje}

\paragraph*{Naloga 1}

S postopkom za hitro urejanje uredite seznam števil:\\
\begin{center}
	1, 5, 2, 6, 3, 7, 4, 5, 8, 7, 6, 5, 6, 3, 6, 5, 4, 8, 7
\end{center}
v padajočem vrstnem redu, pri čemer je delilni element vedno skrajno levi element podtabele, ki se trenutno ureja.

\section{Zunanje urejanje}

\end{document}
